\documentclass[a4paper, 12pt]{article}

\usepackage[left=2cm,right=2cm,
    top=2cm,bottom=2cm,bindingoffset=0cm]{geometry}

\usepackage[T2A]{fontenc}
\usepackage[utf8]{inputenc}
\usepackage{color}
\usepackage{graphicx}

\usepackage{caption}
\usepackage{subcaption}
\usepackage{amssymb}
\usepackage{amsmath}
\usepackage{nicefrac,xfrac}
\usepackage{amsfonts}
\usepackage{mathtools}
\usepackage{color}

\usepackage{booktabs}

\usepackage[dvipsnames]{xcolor}
\usepackage{pgf}
\usepackage{pgfplots}
\usetikzlibrary{calc, patterns, positioning, decorations.pathreplacing}
\usepackage{tikz}
\usepackage[most]{tcolorbox}


\usepackage[english, russian]{babel}
\usepackage{amsmath,amsfonts,amssymb,amsthm,mathtools}

\begin{document}
\begin{center}
    \textbf{Билет №1}
\end{center}
\begin{enumerate}
\item Определение точной верхней грани множества, точной нижней грани множества. Приведите примеры
\item Найти предел последовательности: \[\lim \limits_{n \to +\infty} \frac{1 + 3 + \dots + (2n - 1)}{3n^2}\]
\item Построить график, найти обратную функцию, исследовать свойства четности/нечетности и возрастания/убывания:
$y = 3x^2 - 2x$
\item Найти угол между векторами $\overrightarrow{a}(4; -2; -4)$ и $\overrightarrow{b}(6; -3; 2)$.
\end{enumerate}


\begin{center}
    \textbf{Билет №2}
\end{center}
\begin{enumerate}
\item Определение функции. Определения возрастающей, убывающей, четной, нечетной и периодичной функции. Приведите примеры.
\item Выяснить, ограничены ли последовательности:
\begin{itemize}
    \item \(\displaystyle \frac{4n+7}{6n-3}\)
    \item \((-1)^n \cdot n - \displaystyle \frac{\cos \left ( \displaystyle \frac{\pi n}{4} \right )}{\sqrt 2}\)
\end{itemize}
\item Найти предел функции:
\[\lim \limits_{x \to 1}\ \frac{x^2 - 2x + 1}{x^3 - x^2 - x + 1}\]
\item Найти угол между векторами $\overrightarrow{a}(8; 1; -3)$ и $\overrightarrow{b}(-2; 3; 6)$.
\end{enumerate}

\begin{center}
    \textbf{Билет №3}
\end{center}
\begin{enumerate}
\item Определение последовательности в кванторах. Пример.
\item Найти предел последовательности: \[\lim \limits_{n \to +\infty} \frac{3^n + 5 \cdot 4^{n+1}}{2 \cdot 5^{n-1} + 3^{n+2}}\]
\item Построить график, найти обратную функцию, исследовать свойства четности/нечетности и возрастания/убывания:
$y = \displaystyle \frac{2}{x+4}$
\item Решить систему уравнений методом Крамера
\[
	\begin{cases}
		3x + 2y + 4z = 28 \\
		4x + y + 4z = 27 \\
		4x + 2y + 5z = 34
	\end{cases}
\]
\end{enumerate}

\newpage
\begin{center}
    \textbf{Билет №4}
\end{center}
\begin{enumerate}
\item Определения ограниченной, монотонной последовательности. Примеры.
\item Найти предел последовательности: \[\lim \limits_{n \to +\infty} \frac{(n+4)! - (n+2)!}{(n+3)!}\]
\item Построить график, найти обратную функцию, исследовать свойства четности/нечетности и возрастания/убывания:
$y = 3 \sin(2x)$
\item Найти объем пирамиды, если известны координаты ее вершин:
\[A(2;1;1), B(4;2;3), C(3;4;2), D(3;4;2)\]
\end{enumerate}

\begin{center}
    \textbf{Билет №5}
\end{center}
\begin{enumerate}
\item Определения бесконечно малой и бесконечно больной последовательностей. Сумма бесконечно малых последовательностей. Произведение бесконечно малой и ограниченной последовательностей. Приведите примеры.
\item Найти предел последовательности:  \[\lim \limits_{n \to +\infty} \frac{\sqrt{n+6} - \sqrt{10n - 21}}{5n - 15}\]
\item Найти предел функции: 
\[\lim \limits_{x \to -3}\ \frac{x^3 + 7x^2 + 15x + 9}{x^3 + 8x^2 + 21x + 18}\]
\item Выполнить действия \(A \cdot (B - C)^{-1}\cdot D\), где
\[
		A = \begin{pmatrix}
			-1 & -1 & 2 \\
			1 & 2 & 6 \\
		\end{pmatrix}, \ 
		B = \begin{pmatrix}
			7 & 5 & 3 \\
			2 & 3 & 1 \\
			7 & -1 & 7 \\
		\end{pmatrix}, \ 
		C = \begin{pmatrix}
			4 & 3 & -1 \\
			-2 & 2 & -3 \\
			3 & -3 & 2 \\
		\end{pmatrix}, \ 
		D = \begin{pmatrix}
			-4 & -2 \\
			-5 & 1 \\
			0 & 1 \\
		\end{pmatrix}
	\]
\end{enumerate}

\begin{center}
    \textbf{Билет №6}
\end{center}
\begin{enumerate}
\item Второй замечательный предел последовательности. Число e
\item Найти предел последовательности:  \[\lim \limits_{n \to +\infty} \frac{2n^2 - 3n - 5}{1 + n + 3n^2}\]
\item Найти предел функции: 
\[\lim \limits_{x \to +\infty}\ \frac{(x-1)(x-2)(x-3)(x-4)(x-5)}{(5x-1)^5}\]
\item Найти объем пирамиды, если известны координаты ее вершин:
\[A(-2;4;-2), B(-4;-2;-6), C(6;4;2), D(-6;-4;-2)\]
\end{enumerate}

\begin{center}
    \textbf{Билет №7}
\end{center}
\begin{enumerate}
\item Частичный предел последовательности. Верхний и нижний пределы последовательности. Примеры.
\item Найти частичные пределы последовательности: \[\lim \limits_{n \to +\infty} \frac{(-3)^n + 5 \cdot 4^{n+1}}{2 \cdot (-5)^{n-1} + 3^{n+2}}\]
\item Найти предел функции: 
\[\lim \limits_{x \to 4}\ \frac{\sqrt{1+2x} - 3}{\sqrt{x} - 2}\]
\item Решить систему уравнений методом Гаусса
\[
	\begin{cases}
		x + 2y - 3z - w = -8 \\
		5x + 10y - 16z + w = -39 \\
		x + 2y + 2z - 7w = 11
	\end{cases}
\]
\end{enumerate}

\begin{center}
    \textbf{Билет №8}
\end{center}
\begin{enumerate}
\item Теорема Вейерштрасса о пределе монотонной ограниченной последовательности. 
\item Найти частичные пределы последовательности: \[\lim \limits_{n \to +\infty} n \cdot \cos n\]
\item Найти предел функции: 
\[\lim \limits_{x \to 0}\ \frac{\tg x}{x}\]
\item Решить систему уравнений методом Крамера
\[
	\begin{cases}
		6x + 2y + 7z = 52 \\
		4x + y + 4z = 30 \\
		7x + 2y + 8z = 58
	\end{cases}
\]
\end{enumerate}

\begin{center}
    \textbf{Билет №9}
\end{center}
\begin{enumerate}
\item Предел функции (определение по Коши)
\item Используя метод математической индукции, доказать следующее утверждение:
\[1 + 2 + 2^2 + \dots + 2^{n-1} = 2^n - 1\]
\item Найти производную функции: 
\[y = x \arcsin(\ln x)\]
\item Найти объем пирамиды, если известны координаты ее вершин:
\[A(-3;2;1), B(-6;-3;-5), C(5;6;3), D(-5;-6;-3)\]
\end{enumerate}

\begin{center}
    \textbf{Билет №10}
\end{center}
\begin{enumerate}
\item Предел функции (определение по Гейне). Приведите примеры последовательностей Гейне.
\item Используя метод математической индукции, доказать следующее утверждение:
\[1 + 2 + \cdots + n = \frac{n(n+1)}{2}\]
\item Найти производную функции: 
\[y =  (x^2 - 1)(x^2 - 4)(x^2 + 9)\]
\item Решить систему уравнений методом Крамера
\[
	\begin{cases}
		5x + 3y + 6z = 66 \\
		6x + y + 6z = 59 \\
		6x + 3y + 7z = 75
	\end{cases}
\]
\end{enumerate}

\begin{center}
    \textbf{Билет №11}
\end{center}
\begin{enumerate}
\item Односторонние пределы функции
\item Используя метод математической индукции, доказать следующее утверждение:
\[1^3 + 2^3 + 3^3 + \cdots + n^3 = (1 + 2 + 3 + \cdots + n)^2\]
\item Найти производную функции: 
\[y =  6 \cos \frac{2x}{3}\]
\item Выполнить действия \(A \cdot (B - C)^{-1}\cdot D\), где
\[
		A = \begin{pmatrix}
			-1 & -3 & 4 \\
			1 & 0 & 10 \\
		\end{pmatrix}, \ 
		B = \begin{pmatrix}
			9 & 5 & 3 \\
			4 & 3 & 3 \\
			7 & -1 & 7 \\
		\end{pmatrix}, \ 
		C = \begin{pmatrix}
			6 & 3 & -1 \\
			-2 & 2 & -3 \\
			3 & -3 & 2 \\
		\end{pmatrix}, \ 
		D = \begin{pmatrix}
			-4 & -4 \\
			-9 & 3 \\
			2 & 1 \\
		\end{pmatrix}
	\]
\end{enumerate}

\begin{center}
    \textbf{Билет №12}
\end{center}
\begin{enumerate}
\item Первый замечательный предел (с доказательством)
\item Найти предел последовательности:  
\[\lim \limits_{n \to +\infty} \left ( \frac{1}{4} + \frac{1}{4^2} + \cdots + \frac{1}{4^n} \right )\]
\item Найти производную функции: 
\[y = \frac{(x - 3)^2(2x - 1)}{(x + 1)^3}\]
\item Найти угол между векторами $\overrightarrow{a}(-3; 2; 1)$ и $\overrightarrow{b}(-6; -3; -5)$.
\end{enumerate}

\begin{center}
    \textbf{Билет №13}
\end{center}
\begin{enumerate}
\item Определение производной функции. Производная суммы, произведения и частного двух функций. Примеры
\item Найти предел последовательности:  
\[\lim \limits_{n \to +\infty} \frac{3^{n+1} + 2 \cdot 4^n}{4^{n+1} - 5}\]
\item Найти производную функции: 
\[y = \sqrt[2x]{\frac{x + 2}{x - 2}}\]
\item Решить систему уравнений методом Гаусса
\[
	\begin{cases}
		x + 2y - 3z - w = -11 \\
		6x + 12y - 19z + w = -65 \\
		x + 2y + 2z - 8w = 12
	\end{cases}
\]
\end{enumerate}

\begin{center}
    \textbf{Билет №14}
\end{center}
\begin{enumerate}
\item Производная сложной функции.
Производная функции, заданной параметрически. Примеры
\item Найти предел последовательности:  
\[\lim \limits_{n \to +\infty} \frac{(2n + 1)! + (2n + 2)!}{(2n + 3)!}\]
\item Найти производную функции: 
\[y =  (\ln x)^{\frac{1}{x}}\]
\item Найти объем пирамиды, если известны координаты ее вершин:
\[A(-3;1;2), B(-5;-2;-3), C(3;5;2), D(-3;-5;-2)\]
\end{enumerate}

\begin{center}
    \textbf{Билет №15}
\end{center}
\begin{enumerate}
\item Производные неявной функции
\item Найти предел последовательности:  
\[\lim \limits_{n \to +\infty} (\sqrt{n^2 + 2n + 3} - \sqrt{n^2 - 2n + 5})\]
\item Найти интервалы выпуклости, вогнутости и точки перегиба:
\[y = \frac{x}{1+x^2}\]
\item Найти объем пирамиды, если известны координаты ее вершин:
\[A(-3;4;-1), B(-5;-2;-6), C(6;5;2), D(-6;-5;-2)\]
\end{enumerate}

\begin{center}
    \textbf{Билет №16}
\end{center}
\begin{enumerate}
\item Определение точки экстремума
\item Найти предел последовательности:  
\[\lim \limits_{n \to +\infty} \frac{2n^2 - 3n - 5}{n + 1}\]
\item Найти интервалы выпуклости, вогнутости и точки перегиба:
\[y = \frac{(x-1)^2}{x-2}\]
\item Решить систему уравнений методом Крамера
\[
	\begin{cases}
		6x + 2y + 7z = 59 \\
		5x + y + 5z = 41 \\
		7x + 2y + 8z = 66
	\end{cases}
\]
\end{enumerate}

\begin{center}
    \textbf{Билет №17}
\end{center}
\begin{enumerate}
\item Теорема Лагранжа
\item Найти предел последовательности:  
\[\lim \limits_{n \to +\infty} \frac{7n^3 + 15n^2 + 9n + 1}{5n^4 + 6n^2 - 3n - 4}\]
\item Найти интервалы выпуклости, вогнутости и точки перегиба:
\[y = e^{-x^2}\]
\item Найти угол между векторами $\overrightarrow{a}(-5;-2;-6)$ и $\overrightarrow{b}(-6; -3; -5)$.
\end{enumerate}

\begin{center}
    \textbf{Билет №18}
\end{center}
\begin{enumerate}
\item Правило Лопиталя. Пример использования
\item Найти предел последовательности:  
\[\lim \limits_{n \to +\infty} \frac{2n^2 - 3n + 5}{3n^2 + 4n - 5}\]
\item Даны параметрические уравнения.  Найти \( \displaystyle \frac{dy}{dx}\). 
\[
x = t^2 + 1, \quad y = t^3 - t
\]
\item Найти угол между векторами $\overrightarrow{a}(-5;-2;-6)$ и $\overrightarrow{b}(-3;-5;-2)$.
\end{enumerate}

\newpage
\begin{center}
    \textbf{Билет №19}
\end{center}
\begin{enumerate}
\item Формула Тейлора
\item Найти предел последовательности:  
\[\lim \limits_{n \to +\infty} \frac{(2n + 1)^2 - (n + 1)^2}{n^2 + n + 1}\]
\item Найти асимптоты:
\[y = \frac{5x^2 + 2x -1}{1 - x}\]
\item Выполнить действия \(A \cdot (B - C)^{-1}\cdot D\), где
\[
		A = \begin{pmatrix}
			-2 & -1 & 3 \\
			0 & 3 & 8 \\
		\end{pmatrix}, \ 
		B = \begin{pmatrix}
			9 & 4 & 4 \\
			3 & 3 & 1 \\
			9 & -2 & 8 \\
		\end{pmatrix}, \ 
		C = \begin{pmatrix}
			5 & 4 & -1 \\
			-2 & 2 & -4 \\
			4 & -4 & 2 \\
		\end{pmatrix}, \ 
		D = \begin{pmatrix}
			-6 & -3 \\
			-6 & 1 \\
			1 & 2 \\
		\end{pmatrix}
	\]
\end{enumerate}

\begin{center}
    \textbf{Билет №20}
\end{center}
\begin{enumerate}
\item Линейно зависимая система векторов. Линейно независимая система векторов
\item Найти предел последовательности:  
\[\lim \limits_{n \to +\infty} 1 + \frac{(-1)^n}{n}\]
\item Найти асимптоты
\[y = x + \ln x\]
\item Решить систему уравнений методом Гаусса
\[
	\begin{cases}
		x + 2y - 4z - w = -11 \\
		5x + 10y - 21z + w = -54 \\
		x + 2y + 3z - 7w = 18
	\end{cases}
\]
\end{enumerate}

\begin{center}
    \textbf{Билет №21}
\end{center}
\begin{enumerate}
\item Точка перегиба функции. Выпуклость графика функции
\item Найти частичные пределы последовательности: \[\lim \limits_{n \to +\infty} n \cdot (1 - (-1)^n)\]
\item Найти асимптоты
\[y = \frac{(x+1)^2}{x - 3}\]
\item Найти объем пирамиды, если известны координаты ее вершин:
\[A(1;3;-4), B(-2;-3;-6), C(6;2;3), D(-6;-2;-3)\]
\end{enumerate}

\newpage
\begin{center}
    \textbf{Билет №22}
\end{center}
\begin{enumerate}
\item Наклонная, горизонтальная, вертикальная асимптота 
\item Найти предел последовательности:  
\[\lim \limits_{n \to +\infty} \frac{\sqrt{(n^2 + 5)(n^4 + 2)} - \sqrt{n^6 - 3n^3 +5}}{n}\]
\item Дана неявная функция. Найти её производную.
\[
e^x \sin(y) = x + y
\]
\item Решить систему уравнений методом Гаусса
\[
	\begin{cases}
		x + 2y - 4z - w = -15 \\
		6x + 12y - 25z + w = -89 \\
		x + 2y + 3z - 8w = 20
	\end{cases}
\]
\end{enumerate}

\begin{center}
    \textbf{Билет №23}
\end{center}
\begin{enumerate}
\item Правило Крамера. Пример использования
\item Найти предел последовательности:  
\[\lim \limits_{n \to +\infty} \frac{(n+1)^3 - (n+1)^2}{(n-1)^3 - (n+1)^3}\]
\item Дана неявная функция. Найти её производную.
\[
x^2 + y^2 = 25
\]
\item Решить систему уравнений методом Крамера
\[
	\begin{cases}
		6x + 3y + 7z = 50 \\
		2x + y + 2z = 16 \\
		7x + 3y + 8z = 55
	\end{cases}
\]
\end{enumerate}
\end{document}