\documentclass[a4paper, 12pt]{article}

\usepackage[left=2cm,right=2cm,
    top=2cm,bottom=2cm,bindingoffset=0cm]{geometry}

\usepackage[T2A]{fontenc}
\usepackage[utf8]{inputenc}
\usepackage{color}
\usepackage{graphicx}

\usepackage{caption}
\usepackage{subcaption}
\usepackage{amssymb}
\usepackage{amsmath}
\usepackage{nicefrac,xfrac}
\usepackage{amsfonts}
\usepackage{mathtools}
\usepackage{color}

\usepackage{booktabs}

\usepackage[dvipsnames]{xcolor}
\usepackage{pgf}
\usepackage{pgfplots}
\usetikzlibrary{calc, patterns, positioning, decorations.pathreplacing}
\usepackage{tikz}
\usepackage[most]{tcolorbox}


\usepackage[english, russian]{babel}
\usepackage{amsmath,amsfonts,amssymb,amsthm,mathtools}

\begin{document}
\begin{center}
    \textbf{Билет №1}
\end{center}
\begin{enumerate}
\item Сформулировать критерий Дарбу об интегрируемости функции
\item Найти неопределенный интеграл: $ \displaystyle \int{x\sqrt{1+3x}}dx$
\item Найти определенный интеграл: $\displaystyle \int\limits_{0}^{\ln{2}}{xe^xdx}$
\item Эллипс задан параметрически следующим видом: \\ 

$
	\begin{cases}
		x = a\cdot\cos{t} \\
		y=b\cdot\sin{t} \\
		a > b
		
	\end{cases}
$

Найти длину эллипса в общем виде.

\end{enumerate}

\begin{center}
	\textbf{Билет №2}
\end{center}
\begin{enumerate}
	\item Докажите, что $\displaystyle \int{f^{-1}(x)}dx = x\cdot f(x) - F(f^{-1}(x)) + C$. где $f^{-1}(x)$ - обратная к $f(x)$ функция
	\item Найти неопределенный интеграл: $\displaystyle \int{\frac{dx}{\sqrt{tg(x)}}}$
	\item Найти определенный интеграл: $\displaystyle \int\limits_{0}^{2\pi}{\frac{dx}{1+\varepsilon \cos{x}}; 0 \le \varepsilon < 1}$
	\item $f(x) = xe^x,\ \ \ W(x): f(W(x)) \equiv x$ \\ Найти $\displaystyle \int{W(x)dx}$
	
\end{enumerate}

\begin{center}
	\textbf{Билет №3}
\end{center}
\begin{enumerate}
	\item Определение интеграла по Риману
	\item Найти неопределенный интеграл: $\displaystyle \int{\text{arctg}(x)dx}$
	\item Найти определенный интеграл: $\displaystyle \int\limits_{0}^{e}{\ln{x^2}dx}$
	\item Найти значения $\alpha$, при которых интеграл $\displaystyle \int\limits_{0}^{\inf}{\frac{dx}{x^{\alpha}}}$ имеет конечное значение (сходится).
	
	Подсказка: $\displaystyle \int\limits_{a}^{\inf}{f(x)dx} = \lim_{b\to\inf}{\int_{a}^{b}{f(x)dx}} = \lim_{b\to\inf}{\left(F(b) - F(a)\right)} $
	
\end{enumerate}
\newpage
\begin{center}
	\textbf{Билет №4 без говна}
\end{center}
\begin{enumerate}
	\item Длина дуги в полярных координатах
	\item Найти неопределенный интеграл: $\displaystyle \int{\left(1-\frac{2}{x}\right)^2 e^xdx}$
	\item Найти определенный интеграл: $\displaystyle \int\limits_{0}^{a}{b\sqrt{1 - \frac{x^2}{a^2}}dx} \ ; \ \ a > b$
	\item Найти длину дуги: $\varphi \in [0, \frac{\pi}{4}];\ \  r(\varphi) = \frac{tg(\varphi)}{\cos{\varphi}}$
	
\end{enumerate}

\end{document}