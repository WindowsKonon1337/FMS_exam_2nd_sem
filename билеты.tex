\documentclass[a4paper, 12pt]{article}

\usepackage[left=2cm,right=2cm,
    top=2cm,bottom=2cm,bindingoffset=0cm]{geometry}

\usepackage[T2A]{fontenc}
\usepackage[utf8]{inputenc}
\usepackage{color}
\usepackage{graphicx}

\usepackage{caption}
\usepackage{subcaption}
\usepackage{amssymb}
\usepackage{amsmath}
\usepackage{nicefrac,xfrac}
\usepackage{amsfonts}
\usepackage{mathtools}
\usepackage{color}

\usepackage{booktabs}

\usepackage[dvipsnames]{xcolor}
\usepackage{pgf}
\usepackage{pgfplots}
\usetikzlibrary{calc, patterns, positioning, decorations.pathreplacing}
\usepackage{tikz}
\usepackage[most]{tcolorbox}


\usepackage[english, russian]{babel}
\usepackage{amsmath,amsfonts,amssymb,amsthm,mathtools}

\begin{document}
\begin{center}
    \textbf{Билет №1}
\end{center}
\begin{enumerate}
\item Сформулировать критерий Дарбу об интегрируемости функции
\item Найти неопределенный интеграл: $ \displaystyle \int{x\sqrt{1+3x}}dx$
\item Найти определенный интеграл: $\displaystyle \int\limits_{0}^{\ln{2}}{xe^xdx}$
\item Эллипс задан параметрически следующим видом: \\ 

$
	\begin{cases}
		x = a\cdot\cos{t} \\
		y=b\cdot\sin{t} \\
		a > b
		
	\end{cases}
$

Найти длину эллипса в общем виде.

\end{enumerate}

\begin{center}
	\textbf{Билет №2}
\end{center}
\begin{enumerate}
	\item Длина гладкой кривой, заданной параметрически.
	\item Найти неопределенный интеграл: $\displaystyle \int{\frac{dx}{\sqrt{tg(x)}}}$
	\item Найти определенный интеграл: $\displaystyle \int\limits_{0}^{2\pi}{\frac{dx}{1+\cos{x}}}$
	\item Найти объем фигуры, полученной вращением $\displaystyle f(x) = \frac{1}{x \cdot \ln{x}}$ вокруг оси $OX$. $\displaystyle x \in \left[1,+\infty\right]$
	
\end{enumerate}

\begin{center}
	\textbf{Билет №3}
\end{center}
\begin{enumerate}
	\item Определение интеграла по Риману
	\item Найти неопределенный интеграл: $\displaystyle \int{\text{arctg}(x)dx}$
	\item Найти определенный интеграл: $\displaystyle \int\limits_{0}^{e}{\ln{x^2}dx}$
	\item Найти значения $\alpha$, при которых интеграл $\displaystyle \int\limits_{0}^{+\infty}{\frac{dx}{x^{\alpha}}}$ имеет конечное значение (сходится).
	
	Подсказка: $\displaystyle \int\limits_{a}^{+\infty}{f(x)dx} = \lim_{b\to+\infty}{\int\limits_{a}^{b}{f(x)dx}} = \lim_{b\to+\infty}{\left(F(b) - F(a)\right)} $
	
\end{enumerate}
\newpage
\begin{center}
	\textbf{Билет №4}
\end{center}
\begin{enumerate}
	\item Длина дуги в полярных координатах
	\item Найти неопределенный интеграл: $\displaystyle \int{\left(1-\frac{2}{x}\right)^2 e^xdx}$
	\item Найти определенный интеграл: $\displaystyle \int\limits_{0}^{a}{b\sqrt{1 - \frac{x^2}{a^2}}dx} \ ; \ \ a > b$
	\item Найти длину дуги: $\displaystyle \varphi \in \left[-\frac{\pi}{2}, \frac{\pi}{2}\right];\ \  r(\varphi) = \displaystyle \sqrt{2}e^{\varphi}$
	
\end{enumerate}

\begin{center}
	\textbf{Билет №5}
\end{center}
\begin{enumerate}
	\item Определение первообраной. Определение неопределенного интеграла. Свойства неопределенного интеграла.
	\item Найти неопределенный интеграл: $\displaystyle \int{\frac{dx}{2^x + 1}}$
	\item Найти определенный интеграл: $\displaystyle \int\limits_{0}^{arcsin(\frac{\sqrt{3}}{2})}{\text{arcctg(x)}dx}$
	\item Найти объем тела, полученного вращением $\displaystyle f(x) = e^x$ вокруг оси $OY$. $\displaystyle y \in \left[\ln{2},\ln{32}\right]$ 
\end{enumerate}

\begin{center}
	\textbf{Билет №6}
\end{center}
\begin{enumerate}
	\item Теорема о замене переменной в неопределенном интеграле.
	\item Найти неопределенный интеграл: $\displaystyle \int{e^{\text{arccos}x} dx}$
	\item Найти определенный интеграл: $\displaystyle \int\limits_{-e}^{e}{\text{sh}{x} dx};\ \ \ \  \frac{e^{-x}+e^x}{2} = \text{sh}{x}$
	\item Найти площадь фигуры в полярных координатах: $\displaystyle \varphi \in \left[0, \frac{\pi}{2}\right]; r(\varphi) = 2^\varphi$
\end{enumerate}

\begin{center}
	\textbf{Билет №7}
\end{center}
\begin{enumerate}
	\item Теорема об интегрировании по частям 
	\item Найти неопределенный интеграл: $\displaystyle \int{\frac{dx}{x^2 - 6x}}$
	\item Найти определенный интеграл: $\displaystyle \int\limits_{-e}^{e}{\text{ch}{x}dx}; \ \ \ \ \frac{e^{x} - e^{-x}}{2} = \text{ch}{x}$
	\item Найти объем тела, полученного вращением $\displaystyle f(x) = \arcsin(x)$ вокруг оси $OX$. $\displaystyle x \in \left[0, \frac{\pi}{2}\right]$
\end{enumerate}

\begin{center}
	\textbf{Билет №8}
\end{center}
\begin{enumerate}
	\item Интегрирование тригонометрических функций, подстановки.
	\item Найти неопределенный интеграл: $\displaystyle \int{6 \cdot 3^{x^6 + 2} \cdot x^5 dx}$
	\item Найти определенный интеграл: $\displaystyle \int\limits_{0}^{\pi}{\sin^3{x}\cos^4{x}dx}$
	\item кривая задана в параметрическом виде следующим образом: \\
	$
	\begin{cases}
		x = \cos{t} \\
		y=\sin{t} \\
	\end{cases}
	$
	\\
	Найти длину дуги кривой при $\displaystyle t \in \left[0, \frac{17\pi}{9}\right]$
\end{enumerate}

\begin{center}
	\textbf{Билет №9}
\end{center}
\begin{enumerate}
	\item Задача, приводящая к понятию определенного интеграла.
	\item Найти неопределенный интеграл: $\displaystyle \int{\ln (x + \sqrt{1 + x^2}) dx}$
	\item Найти определенный интеграл: $\displaystyle \int\limits_{1}^{3}{\frac{x+8}{x^2+x+7}dx}$
	\item Найти объем тела, полученного вращением вокруг оси ОХ функции $\displaystyle f(x) = - x ^ {3} + x^2 + x -1; x \in \left[10,100\right]$
\end{enumerate}

\begin{center}
	\textbf{Билет №10}
\end{center}
\begin{enumerate}
	\item Определение разбиения отрезка. Определение интегральной суммы.
	\item Найти неопределенный интеграл: $\displaystyle \int{\frac{x - 3}{\sqrt{x^2 - 6x + 1}} dx}$
	\item Найти определенный интеграл: $\displaystyle \int\limits_{0}^{\pi} \sin^2(x)\,dx$
	\item Найти площадь функции $\displaystyle r(\varphi) = 2(1+\cos(\varphi)); \varphi \in \left[0,\pi\right]$
\end{enumerate}
\newpage
\begin{center}
	\textbf{Билет №11}
\end{center}
\begin{enumerate}
	\item Геомертический смысл определенного интеграла.
	\item Найти неопределенный интеграл: $\displaystyle \int{\frac{2x^2 - 1}{x^3 - 5x^2 +6x} dx}$
	\item Найти определенный интеграл: $\displaystyle \int\limits_{0}^{\pi} \tan^4(x)\,dx$
	\item Найти длину следующей кривой: $\displaystyle r(\varphi) = 4(1+\cos{\varphi}); \varphi \in \left[0, \frac{3\pi}{4}\right]$
\end{enumerate}

\begin{center}
	\textbf{Билет №12}
\end{center}
\begin{enumerate}
	\item Необходимое условие интегрируемости.
	\item Найти неопределенный интеграл: $\displaystyle \int{\frac{\sin^3 x}{\cos^8 x} dx}$
	\item Найти определенный интеграл: $\displaystyle \int\limits_{2}^{3} \frac{dx}{x^2 - 2x - 8}$
	\item Найти объем тела, полученного вращением $\displaystyle f(x)=\log_{7}{e^x}$ вокруг оси ОХ. $x \in \left[0,1\right]$
\end{enumerate}

\begin{center}
	\textbf{Билет №13}
\end{center}
\begin{enumerate}
	\item Верхняя и нижняя суммы Дарбу.
	\item Найти неопределенный интеграл: $\displaystyle \int{\frac{(1 + \cos(2x))^3}{\cos(2x)}dx}$
	\item Найти определенный интеграл: $\displaystyle \int\limits_{4}^{2} \frac{dx}{\sqrt{2 + 3x - 2x^2}}$
	\item Найти длину следующей кривой: $\displaystyle r(\varphi) = \sqrt{3}(1+\sin{\varphi}); \varphi \in \left[0, \frac{\pi}{e}\right]$
\end{enumerate}

\begin{center}
	\textbf{Билет №14}
\end{center}
\begin{enumerate}
	\item Интегралы Дарбу.
	\item Найти неопределенный интеграл: $\displaystyle \int{\sin(10x) \sin(15x) dx}$
	\item Найти определенный интеграл: $\displaystyle \int\limits_0^2 \frac{2x - 1}{2x + 1}\,dx$
	\item Найти объем конуса, радиус основания которого равен $R$, используя определенный интеграл. Высоту конуса считать за $h$. $\displaystyle R > 0, h> 0$
\end{enumerate}
\newpage
\begin{center}
	\textbf{Билет №15}
\end{center}
\begin{enumerate}
	\item Критерий Дарбу интегрируемости функции.
	\item Найти неопределенный интеграл: $\displaystyle \int{\frac{dx}{\sin^2 x \cos^4 x}}$
	\item Найти определенный интеграл: $\displaystyle \int\limits_{0}^{\frac{\pi}{2}} \frac{dx}{3 + 2 \cos(x)}$
	\item Найти объем шара радиуса $R, R > 0$, используя определенный интеграл.
\end{enumerate}

\begin{center}
	\textbf{Билет №16}
\end{center}
\begin{enumerate}
	\item Интегрирование по частям в определенном интеграле.
	\item Найти неопределенный интеграл: $\displaystyle \int{\frac{4 \sin(2x+3)}{\sqrt{4 - 4\cos^2(2x+3)}}dx}$
	\item Найти определенный интеграл: $\displaystyle \int\limits_{-2}^{0} \frac{dx}{\sqrt{x+3}+\sqrt{(x+3)^{3}}}$
	\item Вывести формулу длины окружности, используя определенный интеграл. Считать, что радиус окружности равен $R, R>0$ 
\end{enumerate}

\begin{center}
	\textbf{Билет №17}
\end{center}
\begin{enumerate}
	\item Интеграл от четных функций по симметричному промежутку.
	Интеграл от нечетных функций по симметричному промежутку.
	Интеграл от периодических функций на отрезке, кратном периоду.
	\item Найти неопределенный интеграл: $\displaystyle \int{(x^2 - 2x + 3) \cdot \cos x dx}$
	\item Найти неопределенный интеграл: $\displaystyle \int{(x^2 - 2x + 3) \cdot \cos x dx}$
	\item Вывести формулу площади эллипса, используя определенный интеграл. Эллипс задается следующим образом: $\displaystyle \frac{x^2}{a^2} + \frac{y^2}{b^2} = 1, a > 0, b > 0, a > b$
\end{enumerate}

\begin{center}
	\textbf{Билет №18}
\end{center}
\begin{enumerate}
	\item Определение гладкой кривой. Длина гладкой кривой.
	\item Найти неопределенный интеграл: $\displaystyle \int{\frac{dx}{x \sqrt{1 - 4 \ln x}}}$
	\item Найти определенный интеграл: $\displaystyle \int\limits_{0}^{1} x \arctan(x)\,dx$
	\item Найти объем тела, полученного вращением функции $f(x) = \sin(x^3)$ вокург оси OX. $x\in\left[-\frac{\pi}{4},\frac{\pi}{4}\right]$ 
\end{enumerate}

\end{document}